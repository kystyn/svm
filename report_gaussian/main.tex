\documentclass[14pt,a4paper,article]{ncc}
\usepackage[a4paper, mag=1000, left=2.5cm, right=1cm, top=2cm, bottom=2cm, headsep=0.7cm, footskip=1cm]{geometry}
\usepackage[utf8]{inputenc}
\usepackage[T2A]{fontenc}
\usepackage[english,russian]{babel}
\usepackage{indentfirst}
%\usepackage[dvipsnames]{xcolor}
\usepackage{amsfonts} 
\usepackage{amssymb} 
\usepackage{amsmath, etoolbox}
\usepackage{graphicx}
\usepackage{float}
\graphicspath{figure}
\DeclareGraphicsExtensions{.png,.jpg, .jpeg}

%\bibliographystyle{gost-numeric.bbx}
\usepackage{csquotes}
\usepackage[backend=biber]{biblatex}
\addbibresource{literature.bib}

\usepackage{fancyhdr}
\pagestyle{fancy}
\fancyhead[LE,RO]{\thepage}
\fancyfoot{} 

\usepackage{listings}

%\patchcmd\subequations
%{\theparentequation\alph{equation}}
%{\subequationsformat}
%{}{}

%\newcommand{\subequationsformat}{\theparentequation.\arabic{equation}}

\numberwithin{equation}{subsection}


\usepackage[colorlinks]{hyperref}
\hypersetup{linkcolor=black}

\begin{document}

% Title page 
\begin{titlepage}
    \begin{center}
        \textsc{
            Санкт-Петербургский политехнический университет имени Петра Великого \\[5mm]
            Институт прикладной математики и механики\\[2mm]
            Кафедра прикладной математики
        }   
        \vfill
        \textbf{\large
            Машина опорных векторов\\
            Доклад на тему: \\[3mm]
            ``Асимптотическое поведение LOO-ошибки SV-классификатора с гауссовым ядром''
            %по курсовой работе \\[3mm]
        }                
    \end{center}

    \vfill
    \hfill
    \begin{minipage}{0.5\textwidth}
        Выполнил: \\[2mm]   
		Студент: Дамаскинский Константин \\
		Группа: 3630102/70201\\
    \end{minipage}


    \vfill
    \begin{center}
        \theyear\ г.
    \end{center}
\end{titlepage}

\tableofcontents
\listoffigures
%\listoftables
\newpage

\section{Постановка задачи}

\subsection{Общая формулировка}
Пусть дана тренировочная последовательность $\{x_i, y_i\}, x_i \in \mathbb{R^n}, \; y_i \in \{1, -1\}, \; i \in \overline{1,l}$. Тогда настройка машины опорных векторов состоит в решении следующей задачи оптимизации:

\begin{equation}
\min_{w, b, \xi} \frac{1}{2}w^Tw + C \displaystyle \sum_{i = 1}^{l}\xi_i
\end{equation}
при условии

\begin{align}
y_i(w^T z_i + b) \geq 1 - \xi_i \\
\xi_i \geq 0, i \in \overline{1,l}
\end{align}
где $z_i= \varphi(x_i)$ -- результат отображения тренировочного вектора в пространство размерности $\textrm{dim} w$, $C > 0$ -- штрафной параметр.

К данной задаче квадратичного программирования строится двойственная задача:

\begin{equation}
\min_{\alpha} F(\alpha) = \frac{1}{2}\alpha^TQ\alpha -e^T \alpha
\end{equation}

при условии
\begin{align}
0 \leq \alpha_i \leq C \\
y^T \alpha = 0, i \in \overline{1,l}
\end{align}
где $e=(1, \dots, 1)^T$, $Q$ -- положительно полуопределённая матрица размера $l \times l$, задаваемая по формуле: $Q_{ij}=y_i K(x_i, x_j) y_j, K(x_i, x_j) = \varphi^T(x_i) \varphi(x_j)$ -- \textit{ядро}. Тогда $w= \displaystyle \sum_{i = 1}^{l} \alpha_i y_i \varphi(x_i)$.

В данном докладе мы рассмотрим настройку машины опорных векторов с гауссовским ядром:
\begin{equation}
K(\tilde{x}, \overline{x})= \exp \left( \frac{-\| \tilde{x} - \overline{x} \|^2}{2 \sigma ^2} \right)
\end{equation}

%\subsection{Экспресс-анализ поведения SVM в крайних случаях $C$ и $\sigma$}

%Будем оценивать поведение машины в крайних возможных значениях штрафного параметра $C \; (C \rightarrow 0, C \rightarrow \infty)$ и параметра рассеяния $\sigma \; (\sigma^2 \rightarrow 0, \sigma^2 \rightarrow \infty)$.

%Экспресс-анализ поставленной задачи из соображений здравого смысла позволяет сделать следующие заключения:

%\begin{itemize}
%	\item Сильное недообучение происходит, когда:
%	\begin{enumerate}
%		\item параметр $\sigma$ конечный и $C \rightarrow 0$
%		\item $\sigma \rightarrow 0$ и $C$ -- конечный и достаточно маленький
%		\item $\sigma \rightarrow \infty$ и $C$ -- конечный
%	\end{enumerate}

%	\item Сильное переобучение происходит, когда $\sigma \rightarrow 0$ и $C$ очень большой
%	\item Если $\sigma$ конечный и $C \rightarrow \infty$, то классификатор хорошо разделяет обучающую выборку на два класса, следовательно, в случае зашумлённой обучающей выборки мы получим переобучение
%TODO C = tilde{C} sigma^2
%\end{itemize}

\subsection{LOO-ошибка}

Асимптотическое поведение ошибки обучения SVM можно исследовать путём анализа loo-ошибки (leave-one-out): машину обучают на всей тренировочной последовательности без $i$-го элемента, затем подают на вход $i$-й элемент и проверяют, правильно ли машина его классифицировала. Данную операцию проделывают для всех элементов тренировочной выборки. Доля неверно классифицированных элементов ТП и называется loo-ошибкой.

\printbibliography
%\addcontentsline{toc}{section}{Литература}


\end{document}
